%!TEX TS-program = xelatex
%!TEX encoding = UTF-8 Unicode
% Awesome CV LaTeX Template for CV/Resume
%
% This template has been downloaded from:
% https://github.com/posquit0/Awesome-CV
%
% Author:
% Claud D. Park <posquit0.bj@gmail.com>
% http://www.posquit0.com
%
% Template license:
% CC BY-SA 4.0 (https://creativecommons.org/licenses/by-sa/4.0/)
%


\documentclass[11pt, a4paper]{awesome-cv}
\geometry{left=1.4cm, top=.8cm, right=1.4cm, bottom=1.8cm, footskip=.5cm}
\fontdir[fonts/]

% Color for highlights
% Awesome Colors: awesome-emerald, awesome-skyblue, awesome-red, awesome-pink, awesome-orange
%                 awesome-nephritis, awesome-concrete, awesome-darknight
\colorlet{awesome}{awesome-darknight}
% Uncomment if you would like to specify your own color
% \definecolor{awesome}{HTML}{CA63A8}

% Colors for text
% Uncomment if you would like to specify your own color
% \definecolor{darktext}{HTML}{414141}
% \definecolor{text}{HTML}{333333}
% \definecolor{graytext}{HTML}{5D5D5D}
% \definecolor{lighttext}{HTML}{999999}

% Set false if you don't want to highlight section with awesome color
\setbool{acvSectionColorHighlight}{false}

% If you would like to change the social information separator from a pipe (|) to something else
\renewcommand{\acvHeaderSocialSep}{\quad\textbar\quad}


% Available options: circle|rectangle,edge/noedge,left/right
% \photo{./profile.png}
\name{Ankur}{Singh}
\position{Senior Undergraduate{\enskip\cdotp\enskip}Electrical Engineering}
\address{Indian Institute of Technology, Kanpur}
\mobile{(+91) 900-594-1110}
\email{ankuriit@iitk.ac.in \quad\textbar\quad singh.ankur33214@gmail.com}
\homepage{ankur219.github.io}
\github{ankur219}
\linkedin{ankur219}
% \twitter{@therealyashsriv}
% \quote{``There is no fate but what we make."}

\newcommand{\smallcventry}[6]{\cventry{#1}{#2}{#3}{#4}{#6}}
\newcommand{\specialcvsection}[1]{\cvsection{#1}}




\begin{document}
\makecvheader
\makecvfooter
  {}
  {}
  {\thepage}

\specialcvsection{Educational Qualifications}

\newcommand{\education}[4]{
  & #1 & #2 & &#3 & #4
}
\begin{tabular}{ | L{0.05cm} l | L{3cm} | L{0.05cm} C{8cm} | r |}
  \hline
  \education{\textbf{Year}}{\textbf{Degree}}{\textbf{Institution(Board)}}\\
  \hline
  \education{2016 -- Present}{B.Tech, Electrical Engineering{Indian Institute of Technology, Kanpur}\\
  \education{2015}{AISSCE -- XII}{Air Force School, Viman Nagar, Pune}\\
  \education{2013}{AISSE -- X}{Air Force School, Viman Nagar, Pune}\\
  \hline
\end{tabular}

%%% Local Variables:
%%% mode: latex
%%% TeX-master: "../cv"
%%% TeX-engine: xelatex
%%% End:

\cvsection{Scholastic Achievements}
\begin{cvhonors}

  \cvhonor
  {All India Rank 636}
  {Joint Entrance Exam Advanced}
  {India}
  {2016}

  

\end{cvhonors}

%%% Local Variables:
%%% mode: latex
%%% TeX-engine: xetex
%%% TeX-master: "../cv"
%%% End:

\input{sections/experience.tex}
\input{sections/project.tex}
\input{sections/coursework.tex}
\input{sections/positions.tex}
\input{sections/skills.tex}
\input{sections/misc.tex}
% \input{sections/interests.tex}


\end{document}

%%% Local Variables:
%%% mode: latex
%%% TeX-engine: xetex
%%% End:
